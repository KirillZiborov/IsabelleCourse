%
\begin{isabellebody}%
\setisabellecontext{Intro}%
%
\isadelimdocument
%
\endisadelimdocument
%
\isatagdocument
%
\isamarkupsection{Introduction to Isabelle%
}
\isamarkuptrue%
%
\endisatagdocument
{\isafolddocument}%
%
\isadelimdocument
%
\endisadelimdocument
%
\isadelimtheory
%
\endisadelimtheory
%
\isatagtheory
%
\endisatagtheory
{\isafoldtheory}%
%
\isadelimtheory
%
\endisadelimtheory
%
\begin{isamarkuptext}%
This is also a comment but it generates nice \LaTeX-text!%
\end{isamarkuptext}\isamarkuptrue%
%
\begin{isamarkuptext}%
Command \isacommand{thy{\isacharunderscore}{\kern0pt}deps} demonstrates a graph of dependecies between Isabelle/HOL theories:%
\end{isamarkuptext}\isamarkuptrue%
%
\isadelimdocument
%
\endisadelimdocument
%
\isatagdocument
%
\isamarkupsubsection{Terms \isa{{\isacharampersand}{\kern0pt}} Types%
}
\isamarkuptrue%
%
\endisatagdocument
{\isafolddocument}%
%
\isadelimdocument
%
\endisadelimdocument
%
\begin{isamarkuptext}%
Note that free variables (eg \isa{x}), 
 bound variables (eg \isa{{\isasymlambda}n{\isachardot}{\kern0pt}\ n}) and
 constants (eg \isa{Suc}) are displayed differently.%
\end{isamarkuptext}\isamarkuptrue%
\isacommand{term}\isamarkupfalse%
\ {\isachardoublequoteopen}x{\isachardoublequoteclose}\isanewline
\isacommand{term}\isamarkupfalse%
\ {\isachardoublequoteopen}Suc\ x{\isachardoublequoteclose}\ \isanewline
\isacommand{term}\isamarkupfalse%
\ {\isachardoublequoteopen}Succ\ x{\isachardoublequoteclose}\isanewline
\isacommand{term}\isamarkupfalse%
\ {\isachardoublequoteopen}Suc\ x\ {\isacharequal}{\kern0pt}\ Succ\ y{\isachardoublequoteclose}\isanewline
\isacommand{term}\isamarkupfalse%
\ {\isachardoublequoteopen}{\isasymlambda}x{\isachardot}{\kern0pt}\ x{\isachardoublequoteclose}%
\begin{isamarkuptext}%
A bound variable:%
\end{isamarkuptext}\isamarkuptrue%
\isacommand{term}\isamarkupfalse%
\ {\isachardoublequoteopen}{\isasymlambda}x{\isachardot}{\kern0pt}\ Suc\ x\ {\isacharless}{\kern0pt}\ y{\isachardoublequoteclose}\isanewline
\isacommand{prop}\isamarkupfalse%
\ {\isachardoublequoteopen}map\ {\isacharparenleft}{\kern0pt}{\isasymlambda}n{\isachardot}{\kern0pt}\ Suc\ n\ {\isacharplus}{\kern0pt}\ {\isadigit{1}}{\isacharparenright}{\kern0pt}\ {\isacharbrackleft}{\kern0pt}{\isadigit{0}}{\isacharcomma}{\kern0pt}\ {\isadigit{1}}{\isacharbrackright}{\kern0pt}\ {\isacharequal}{\kern0pt}\ {\isacharbrackleft}{\kern0pt}{\isadigit{2}}{\isacharcomma}{\kern0pt}\ {\isadigit{3}}{\isacharbrackright}{\kern0pt}{\isachardoublequoteclose}%
\begin{isamarkuptext}%
To display types inside terms:%
\end{isamarkuptext}\isamarkuptrue%
\isacommand{declare}\isamarkupfalse%
\ {\isacharbrackleft}{\kern0pt}{\isacharbrackleft}{\kern0pt}show{\isacharunderscore}{\kern0pt}types{\isacharbrackright}{\kern0pt}{\isacharbrackright}{\kern0pt}\isanewline
\isacommand{term}\isamarkupfalse%
\ {\isachardoublequoteopen}Suc\ x\ {\isacharequal}{\kern0pt}\ Succ\ y{\isachardoublequoteclose}%
\begin{isamarkuptext}%
To switch off again:%
\end{isamarkuptext}\isamarkuptrue%
\isacommand{declare}\isamarkupfalse%
\ {\isacharbrackleft}{\kern0pt}{\isacharbrackleft}{\kern0pt}show{\isacharunderscore}{\kern0pt}types{\isacharequal}{\kern0pt}false{\isacharbrackright}{\kern0pt}{\isacharbrackright}{\kern0pt}\isanewline
\isacommand{term}\isamarkupfalse%
\ {\isachardoublequoteopen}Suc\ x\ {\isacharequal}{\kern0pt}\ Succ\ y{\isachardoublequoteclose}%
\begin{isamarkuptext}%
Numbers and + are overloaded:%
\end{isamarkuptext}\isamarkuptrue%
\isacommand{prop}\isamarkupfalse%
\ {\isachardoublequoteopen}n\ {\isacharplus}{\kern0pt}\ n\ {\isacharequal}{\kern0pt}\ {\isadigit{0}}{\isachardoublequoteclose}\isanewline
\isacommand{prop}\isamarkupfalse%
\ {\isachardoublequoteopen}{\isacharparenleft}{\kern0pt}n{\isacharcolon}{\kern0pt}{\isacharcolon}{\kern0pt}nat{\isacharparenright}{\kern0pt}\ {\isacharplus}{\kern0pt}\ n\ {\isacharequal}{\kern0pt}\ {\isadigit{0}}{\isachardoublequoteclose}\isanewline
\isacommand{prop}\isamarkupfalse%
\ {\isachardoublequoteopen}{\isacharparenleft}{\kern0pt}n{\isacharcolon}{\kern0pt}{\isacharcolon}{\kern0pt}int{\isacharparenright}{\kern0pt}\ {\isacharplus}{\kern0pt}\ n\ {\isacharequal}{\kern0pt}\ {\isadigit{0}}{\isachardoublequoteclose}\isanewline
\isacommand{prop}\isamarkupfalse%
\ {\isachardoublequoteopen}n\ {\isacharplus}{\kern0pt}\ n\ {\isacharequal}{\kern0pt}\ Suc\ m{\isachardoublequoteclose}%
\begin{isamarkuptext}%
Terms must be type correct! Try this: \texttt{term "True + False"}%
\end{isamarkuptext}\isamarkuptrue%
%
\begin{isamarkuptext}%
Displaying theorems, schematic variables%
\end{isamarkuptext}\isamarkuptrue%
\isacommand{thm}\isamarkupfalse%
\ conjI%
\begin{isamarkuptext}%
Schematic variables have question marks and can be instantiated:%
\end{isamarkuptext}\isamarkuptrue%
\isacommand{thm}\isamarkupfalse%
\ conjI\ {\isacharbrackleft}{\kern0pt}\isakeyword{where}\ {\isacharquery}{\kern0pt}Q\ {\isacharequal}{\kern0pt}\ {\isachardoublequoteopen}x{\isachardoublequoteclose}{\isacharbrackright}{\kern0pt}\isanewline
\isacommand{thm}\isamarkupfalse%
\ conjI\ {\isacharbrackleft}{\kern0pt}no{\isacharunderscore}{\kern0pt}vars{\isacharbrackright}{\kern0pt}\isanewline
\isacommand{thm}\isamarkupfalse%
\ impI\isanewline
\isacommand{thm}\isamarkupfalse%
\ conjE%
\begin{isamarkuptext}%
You can use \texttt{term}, \texttt{prop} and \texttt{thm} in \LaTeX
  sections, too!  The lemma conjI is: \isa{{\isasymlbrakk}{\isacharquery}{\kern0pt}P{\isacharsemicolon}{\kern0pt}\ {\isacharquery}{\kern0pt}Q{\isasymrbrakk}\ {\isasymLongrightarrow}\ {\isacharquery}{\kern0pt}P\ {\isasymand}\ {\isacharquery}{\kern0pt}Q}. Nicer version, 
  without schematic variables: \isa{{\isasymlbrakk}P{\isacharsemicolon}{\kern0pt}\ Q{\isasymrbrakk}\ {\isasymLongrightarrow}\ P\ {\isasymand}\ Q}.%
\end{isamarkuptext}\isamarkuptrue%
%
\begin{isamarkuptext}%
Finding theorems%
\end{isamarkuptext}\isamarkuptrue%
%
\begin{isamarkuptext}%
Searching for constants/functions:%
\end{isamarkuptext}\isamarkuptrue%
\isacommand{find{\isacharunderscore}{\kern0pt}theorems}\isamarkupfalse%
\ {\isachardoublequoteopen}map{\isachardoublequoteclose}%
\begin{isamarkuptext}%
A list of search criteria finds thms that match all of them:%
\end{isamarkuptext}\isamarkuptrue%
\isacommand{find{\isacharunderscore}{\kern0pt}theorems}\isamarkupfalse%
\ {\isachardoublequoteopen}map{\isachardoublequoteclose}\ {\isachardoublequoteopen}zip{\isachardoublequoteclose}%
\begin{isamarkuptext}%
To search for patterns, use underscore:%
\end{isamarkuptext}\isamarkuptrue%
\isacommand{find{\isacharunderscore}{\kern0pt}theorems}\isamarkupfalse%
\ {\isachardoublequoteopen}{\isacharunderscore}{\kern0pt}\ {\isacharplus}{\kern0pt}\ {\isacharunderscore}{\kern0pt}\ {\isacharequal}{\kern0pt}\ {\isacharunderscore}{\kern0pt}\ {\isacharminus}{\kern0pt}\ {\isacharunderscore}{\kern0pt}{\isachardoublequoteclose}\isanewline
\isacommand{find{\isacharunderscore}{\kern0pt}theorems}\isamarkupfalse%
\ {\isachardoublequoteopen}{\isacharunderscore}{\kern0pt}\ {\isacharplus}{\kern0pt}\ {\isacharunderscore}{\kern0pt}{\isachardoublequoteclose}\ {\isachardoublequoteopen}{\isacharunderscore}{\kern0pt}\ {\isacharless}{\kern0pt}\ {\isacharunderscore}{\kern0pt}{\isachardoublequoteclose}\ {\isachardoublequoteopen}Suc{\isachardoublequoteclose}\isanewline
\isacommand{find{\isacharunderscore}{\kern0pt}theorems}\isamarkupfalse%
\ {\isachardoublequoteopen}Suc\ {\isacharparenleft}{\kern0pt}Suc\ {\isacharunderscore}{\kern0pt}{\isacharparenright}{\kern0pt}{\isachardoublequoteclose}%
\begin{isamarkuptext}%
Searching for theorem names:%
\end{isamarkuptext}\isamarkuptrue%
\isacommand{find{\isacharunderscore}{\kern0pt}theorems}\isamarkupfalse%
\ name{\isacharcolon}{\kern0pt}\ {\isachardoublequoteopen}conj{\isachardoublequoteclose}%
\begin{isamarkuptext}%
They can all be combined, theorem names include the theory name:%
\end{isamarkuptext}\isamarkuptrue%
\isacommand{find{\isacharunderscore}{\kern0pt}theorems}\isamarkupfalse%
\ {\isachardoublequoteopen}{\isacharunderscore}{\kern0pt}\ {\isasymand}\ {\isacharunderscore}{\kern0pt}{\isachardoublequoteclose}\ name{\isacharcolon}{\kern0pt}\ {\isachardoublequoteopen}HOL{\isachardot}{\kern0pt}{\isachardoublequoteclose}\ {\isacharminus}{\kern0pt}name{\isacharcolon}{\kern0pt}\ {\isachardoublequoteopen}conj{\isachardoublequoteclose}%
\isadelimdocument
%
\endisadelimdocument
%
\isatagdocument
%
\isamarkupsubsection{Basic proofs%
}
\isamarkuptrue%
%
\endisatagdocument
{\isafolddocument}%
%
\isadelimdocument
%
\endisadelimdocument
%
\begin{isamarkuptext}%
Stating theorems and a first proof%
\end{isamarkuptext}\isamarkuptrue%
\isacommand{lemma}\isamarkupfalse%
\ {\isachardoublequoteopen}A\ {\isasymlongrightarrow}\ A{\isachardoublequoteclose}\isanewline
%
\isadelimproof
\ \ %
\endisadelimproof
%
\isatagproof
\isacommand{apply}\isamarkupfalse%
\ {\isacharparenleft}{\kern0pt}rule\ impI{\isacharparenright}{\kern0pt}\ \ \isanewline
\ \ \isacommand{apply}\isamarkupfalse%
\ assumption\isanewline
\ \ \isacommand{done}\isamarkupfalse%
%
\endisatagproof
{\isafoldproof}%
%
\isadelimproof
%
\endisadelimproof
%
\begin{isamarkuptext}%
A proof is a list of {\tt apply} statements, terminated by {\tt done}.

  {\tt apply} takes a proof method as argument (assumption, rule,
  etc.). It needs parentheses when the method consists of more than
  one word.%
\end{isamarkuptext}\isamarkuptrue%
%
\begin{isamarkuptext}%
Isabelle doesn't care if you call it lemma, theorem or corollary%
\end{isamarkuptext}\isamarkuptrue%
\isacommand{theorem}\isamarkupfalse%
\ {\isachardoublequoteopen}A\ {\isasymlongrightarrow}\ A{\isachardoublequoteclose}\ \isanewline
%
\isadelimproof
\ \ %
\endisadelimproof
%
\isatagproof
\isacommand{apply}\isamarkupfalse%
\ {\isacharparenleft}{\kern0pt}rule\ impI{\isacharparenright}{\kern0pt}\isanewline
\ \ \isacommand{apply}\isamarkupfalse%
\ assumption\isanewline
\ \ \isacommand{done}\isamarkupfalse%
%
\endisatagproof
{\isafoldproof}%
%
\isadelimproof
\isanewline
%
\endisadelimproof
\isanewline
\isacommand{corollary}\isamarkupfalse%
\ {\isachardoublequoteopen}A\ {\isasymlongrightarrow}\ A{\isachardoublequoteclose}\ \isanewline
%
\isadelimproof
\ \ %
\endisadelimproof
%
\isatagproof
\isacommand{apply}\isamarkupfalse%
\ {\isacharparenleft}{\kern0pt}rule\ impI{\isacharparenright}{\kern0pt}\isanewline
\ \ \isacommand{apply}\isamarkupfalse%
\ assumption\isanewline
\ \ \isacommand{done}\isamarkupfalse%
%
\endisatagproof
{\isafoldproof}%
%
\isadelimproof
%
\endisadelimproof
%
\begin{isamarkuptext}%
You can give it a name%
\end{isamarkuptext}\isamarkuptrue%
\isacommand{lemma}\isamarkupfalse%
\ mylemma{\isacharcolon}{\kern0pt}\ {\isachardoublequoteopen}A\ {\isasymlongrightarrow}\ A{\isachardoublequoteclose}%
\isadelimproof
\ %
\endisadelimproof
%
\isatagproof
\isacommand{by}\isamarkupfalse%
\ {\isacharparenleft}{\kern0pt}rule\ impI{\isacharparenright}{\kern0pt}%
\endisatagproof
{\isafoldproof}%
%
\isadelimproof
%
\endisadelimproof
%
\begin{isamarkuptext}%
Abandoning a proof%
\end{isamarkuptext}\isamarkuptrue%
\isacommand{lemma}\isamarkupfalse%
\ {\isachardoublequoteopen}P\ {\isacharequal}{\kern0pt}\ NP{\isachardoublequoteclose}\isanewline
\ \ %
\isamarkupcmt{this is too hard%
}\isanewline
%
\isadelimproof
\ \ %
\endisadelimproof
%
\isatagproof
\isacommand{oops}\isamarkupfalse%
%
\endisatagproof
{\isafoldproof}%
%
\isadelimproof
%
\endisadelimproof
%
\begin{isamarkuptext}%
Isabelle forgets the lemma and you cannot use it later%
\end{isamarkuptext}\isamarkuptrue%
%
\begin{isamarkuptext}%
Faking a proof%
\end{isamarkuptext}\isamarkuptrue%
\isacommand{lemma}\isamarkupfalse%
\ name{\isadigit{1}}{\isacharcolon}{\kern0pt}\ {\isachardoublequoteopen}P\ {\isasymnoteq}\ NP{\isachardoublequoteclose}\isanewline
\ \ %
\isamarkupcmt{have an idea, will show this later%
}\isanewline
%
\isadelimproof
\ \ %
\endisadelimproof
%
\isatagproof
\isacommand{sorry}\isamarkupfalse%
%
\endisatagproof
{\isafoldproof}%
%
\isadelimproof
%
\endisadelimproof
%
\begin{isamarkuptext}%
{\tt sorry} only works interactively, 
  and Isabelle keeps track of what you have faked.%
\end{isamarkuptext}\isamarkuptrue%
%
\begin{isamarkuptext}%
Proof styles%
\end{isamarkuptext}\isamarkuptrue%
\isacommand{theorem}\isamarkupfalse%
\ Cantor{\isacharcolon}{\kern0pt}\ {\isachardoublequoteopen}{\isasymexists}S{\isachardot}{\kern0pt}\ S\ {\isasymnotin}\ range\ {\isacharparenleft}{\kern0pt}f\ {\isacharcolon}{\kern0pt}{\isacharcolon}{\kern0pt}\ {\isacharprime}{\kern0pt}a\ {\isasymRightarrow}\ {\isacharprime}{\kern0pt}a\ set{\isacharparenright}{\kern0pt}{\isachardoublequoteclose}%
\isadelimproof
\ %
\endisadelimproof
%
\isatagproof
\isacommand{by}\isamarkupfalse%
\ best\isanewline
\isanewline
%
\isamarkupcmt{exploring, but unstructured%
}%
\endisatagproof
{\isafoldproof}%
%
\isadelimproof
%
\endisadelimproof
\isanewline
\isacommand{theorem}\isamarkupfalse%
\ Cantor{\isacharprime}{\kern0pt}{\isacharcolon}{\kern0pt}\ {\isachardoublequoteopen}{\isasymexists}S{\isachardot}{\kern0pt}\ S\ {\isasymnotin}\ range\ {\isacharparenleft}{\kern0pt}f\ {\isacharcolon}{\kern0pt}{\isacharcolon}{\kern0pt}\ {\isacharprime}{\kern0pt}a\ {\isasymRightarrow}\ {\isacharprime}{\kern0pt}a\ set{\isacharparenright}{\kern0pt}{\isachardoublequoteclose}\ \isanewline
%
\isadelimproof
\ \ %
\endisadelimproof
%
\isatagproof
\isacommand{apply}\isamarkupfalse%
\ {\isacharparenleft}{\kern0pt}rule{\isacharunderscore}{\kern0pt}tac\ x\ {\isacharequal}{\kern0pt}\ {\isachardoublequoteopen}{\isacharbraceleft}{\kern0pt}x{\isachardot}{\kern0pt}\ x\ {\isasymnotin}\ f\ x{\isacharbraceright}{\kern0pt}{\isachardoublequoteclose}\ \isakeyword{in}\ exI{\isacharparenright}{\kern0pt}\isanewline
\ \ \isacommand{apply}\isamarkupfalse%
\ {\isacharparenleft}{\kern0pt}rule\ notI{\isacharparenright}{\kern0pt}\ \isanewline
\ \ \isacommand{apply}\isamarkupfalse%
\ clarsimp\isanewline
\ \ \isacommand{apply}\isamarkupfalse%
\ blast\isanewline
\ \ \isacommand{done}\isamarkupfalse%
\ \ \ \ \isanewline
\isanewline
%
\isamarkupcmt{structured, explaining%
}%
\endisatagproof
{\isafoldproof}%
%
\isadelimproof
\isanewline
%
\endisadelimproof
\isacommand{theorem}\isamarkupfalse%
\ Cantor{\isacharprime}{\kern0pt}{\isacharprime}{\kern0pt}{\isacharcolon}{\kern0pt}\ {\isachardoublequoteopen}{\isasymexists}S{\isachardot}{\kern0pt}\ S\ {\isasymnotin}\ range\ {\isacharparenleft}{\kern0pt}f\ {\isacharcolon}{\kern0pt}{\isacharcolon}{\kern0pt}\ {\isacharprime}{\kern0pt}a\ {\isasymRightarrow}\ {\isacharprime}{\kern0pt}a\ set{\isacharparenright}{\kern0pt}{\isachardoublequoteclose}\isanewline
%
\isadelimproof
%
\endisadelimproof
%
\isatagproof
\isacommand{proof}\isamarkupfalse%
\isanewline
\ \ \isacommand{let}\isamarkupfalse%
\ {\isacharquery}{\kern0pt}S\ {\isacharequal}{\kern0pt}\ {\isachardoublequoteopen}{\isacharbraceleft}{\kern0pt}x{\isachardot}{\kern0pt}\ x\ {\isasymnotin}\ f\ x{\isacharbraceright}{\kern0pt}{\isachardoublequoteclose}\isanewline
\ \ \isacommand{show}\isamarkupfalse%
\ {\isachardoublequoteopen}{\isacharquery}{\kern0pt}S\ {\isasymnotin}\ range\ f{\isachardoublequoteclose}\isanewline
\ \ \isacommand{proof}\isamarkupfalse%
\isanewline
\ \ \ \ \isacommand{assume}\isamarkupfalse%
\ {\isachardoublequoteopen}{\isacharquery}{\kern0pt}S\ {\isasymin}\ range\ f{\isachardoublequoteclose}\isanewline
\ \ \ \ \isacommand{then}\isamarkupfalse%
\ \isacommand{obtain}\isamarkupfalse%
\ y\ \isakeyword{where}\ fy{\isacharcolon}{\kern0pt}\ {\isachardoublequoteopen}{\isacharquery}{\kern0pt}S\ {\isacharequal}{\kern0pt}\ f\ y{\isachardoublequoteclose}\ \isacommand{{\isachardot}{\kern0pt}{\isachardot}{\kern0pt}}\isamarkupfalse%
\isanewline
\ \ \ \ \isacommand{show}\isamarkupfalse%
\ False\isanewline
\ \ \ \ \isacommand{proof}\isamarkupfalse%
\ cases\isanewline
\ \ \ \ \ \ \isacommand{assume}\isamarkupfalse%
\ yin{\isacharcolon}{\kern0pt}\ {\isachardoublequoteopen}y\ {\isasymin}\ {\isacharquery}{\kern0pt}S{\isachardoublequoteclose}\isanewline
\ \ \ \ \ \ \isacommand{hence}\isamarkupfalse%
\ {\isachardoublequoteopen}y\ {\isasymnotin}\ f\ y{\isachardoublequoteclose}\ \isacommand{by}\isamarkupfalse%
\ simp\isanewline
\ \ \ \ \ \ \isacommand{hence}\isamarkupfalse%
\ {\isachardoublequoteopen}y\ {\isasymnotin}\ {\isacharquery}{\kern0pt}S{\isachardoublequoteclose}\ \ \isacommand{by}\isamarkupfalse%
{\isacharparenleft}{\kern0pt}simp\ add{\isacharcolon}{\kern0pt}fy{\isacharparenright}{\kern0pt}\isanewline
\ \ \ \ \ \ \isacommand{thus}\isamarkupfalse%
\ False\ \isacommand{using}\isamarkupfalse%
\ yin\ \isacommand{by}\isamarkupfalse%
\ contradiction\isanewline
\ \ \ \ \isacommand{next}\isamarkupfalse%
\isanewline
\ \ \ \ \ \ \isacommand{assume}\isamarkupfalse%
\ yout{\isacharcolon}{\kern0pt}\ {\isachardoublequoteopen}y\ {\isasymnotin}\ {\isacharquery}{\kern0pt}S{\isachardoublequoteclose}\isanewline
\ \ \ \ \ \ \isacommand{hence}\isamarkupfalse%
\ {\isachardoublequoteopen}y\ {\isasymin}\ f\ y{\isachardoublequoteclose}\ \isacommand{by}\isamarkupfalse%
\ simp\isanewline
\ \ \ \ \ \ \isacommand{hence}\isamarkupfalse%
\ {\isachardoublequoteopen}y\ {\isasymin}\ {\isacharquery}{\kern0pt}S{\isachardoublequoteclose}\ \ \isacommand{by}\isamarkupfalse%
{\isacharparenleft}{\kern0pt}simp\ add{\isacharcolon}{\kern0pt}fy{\isacharparenright}{\kern0pt}\isanewline
\ \ \ \ \ \ \isacommand{thus}\isamarkupfalse%
\ False\ \isacommand{using}\isamarkupfalse%
\ yout\ \isacommand{by}\isamarkupfalse%
\ contradiction\isanewline
\ \ \ \ \isacommand{qed}\isamarkupfalse%
\isanewline
\ \ \isacommand{qed}\isamarkupfalse%
\isanewline
\isacommand{qed}\isamarkupfalse%
%
\endisatagproof
{\isafoldproof}%
%
\isadelimproof
\isanewline
%
\endisadelimproof
%
\isadelimtheory
\isanewline
%
\endisadelimtheory
%
\isatagtheory
\isacommand{end}\isamarkupfalse%
%
\endisatagtheory
{\isafoldtheory}%
%
\isadelimtheory
%
\endisadelimtheory
%
\end{isabellebody}%
\endinput
%:%file=Intro.tex%:%
%:%11=1%:%
%:%36=9%:%
%:%40=11%:%
%:%49=13%:%
%:%61=15%:%
%:%62=16%:%
%:%63=17%:%
%:%65=20%:%
%:%66=20%:%
%:%67=21%:%
%:%68=21%:%
%:%69=22%:%
%:%70=22%:%
%:%71=23%:%
%:%72=23%:%
%:%73=24%:%
%:%74=24%:%
%:%76=26%:%
%:%78=27%:%
%:%79=27%:%
%:%80=28%:%
%:%81=28%:%
%:%83=30%:%
%:%85=31%:%
%:%86=31%:%
%:%87=32%:%
%:%88=32%:%
%:%90=34%:%
%:%92=35%:%
%:%93=35%:%
%:%94=36%:%
%:%95=36%:%
%:%97=38%:%
%:%99=40%:%
%:%100=40%:%
%:%101=41%:%
%:%102=41%:%
%:%103=42%:%
%:%104=42%:%
%:%105=43%:%
%:%106=43%:%
%:%108=45%:%
%:%112=47%:%
%:%114=49%:%
%:%115=49%:%
%:%117=50%:%
%:%119=51%:%
%:%120=51%:%
%:%121=52%:%
%:%122=52%:%
%:%123=53%:%
%:%124=53%:%
%:%125=54%:%
%:%126=54%:%
%:%128=57%:%
%:%129=58%:%
%:%130=59%:%
%:%134=62%:%
%:%138=64%:%
%:%140=65%:%
%:%141=65%:%
%:%143=67%:%
%:%145=68%:%
%:%146=68%:%
%:%148=70%:%
%:%150=71%:%
%:%151=71%:%
%:%152=72%:%
%:%153=72%:%
%:%154=73%:%
%:%155=73%:%
%:%157=75%:%
%:%159=76%:%
%:%160=76%:%
%:%162=78%:%
%:%164=79%:%
%:%165=79%:%
%:%172=81%:%
%:%184=82%:%
%:%186=84%:%
%:%187=84%:%
%:%190=85%:%
%:%194=85%:%
%:%195=85%:%
%:%196=86%:%
%:%197=86%:%
%:%198=87%:%
%:%208=90%:%
%:%209=91%:%
%:%210=92%:%
%:%211=93%:%
%:%212=94%:%
%:%216=97%:%
%:%218=99%:%
%:%219=99%:%
%:%222=100%:%
%:%226=100%:%
%:%227=100%:%
%:%228=101%:%
%:%229=101%:%
%:%230=102%:%
%:%236=102%:%
%:%239=103%:%
%:%240=104%:%
%:%241=104%:%
%:%244=105%:%
%:%248=105%:%
%:%249=105%:%
%:%250=106%:%
%:%251=106%:%
%:%252=107%:%
%:%262=109%:%
%:%264=111%:%
%:%265=111%:%
%:%267=111%:%
%:%271=111%:%
%:%272=111%:%
%:%281=113%:%
%:%283=115%:%
%:%284=115%:%
%:%285=116%:%
%:%286=116%:%
%:%287=116%:%
%:%290=117%:%
%:%294=117%:%
%:%304=119%:%
%:%308=121%:%
%:%310=123%:%
%:%311=123%:%
%:%312=124%:%
%:%313=124%:%
%:%314=124%:%
%:%317=125%:%
%:%321=125%:%
%:%331=128%:%
%:%332=129%:%
%:%336=132%:%
%:%338=135%:%
%:%339=135%:%
%:%341=135%:%
%:%345=135%:%
%:%346=135%:%
%:%347=136%:%
%:%349=137%:%
%:%357=137%:%
%:%358=138%:%
%:%359=138%:%
%:%362=139%:%
%:%366=139%:%
%:%367=139%:%
%:%368=140%:%
%:%369=140%:%
%:%370=141%:%
%:%371=141%:%
%:%372=142%:%
%:%373=142%:%
%:%374=143%:%
%:%375=143%:%
%:%376=144%:%
%:%378=145%:%
%:%384=145%:%
%:%387=146%:%
%:%388=146%:%
%:%395=147%:%
%:%396=147%:%
%:%397=148%:%
%:%398=148%:%
%:%399=149%:%
%:%400=149%:%
%:%401=150%:%
%:%402=150%:%
%:%403=151%:%
%:%404=151%:%
%:%405=152%:%
%:%406=152%:%
%:%407=152%:%
%:%408=152%:%
%:%409=153%:%
%:%410=153%:%
%:%411=154%:%
%:%412=154%:%
%:%413=155%:%
%:%414=155%:%
%:%415=156%:%
%:%416=156%:%
%:%417=156%:%
%:%418=157%:%
%:%419=157%:%
%:%420=157%:%
%:%421=158%:%
%:%422=158%:%
%:%423=158%:%
%:%424=158%:%
%:%425=159%:%
%:%426=159%:%
%:%427=160%:%
%:%428=160%:%
%:%429=161%:%
%:%430=161%:%
%:%431=161%:%
%:%432=162%:%
%:%433=162%:%
%:%434=162%:%
%:%435=163%:%
%:%436=163%:%
%:%437=163%:%
%:%438=163%:%
%:%439=164%:%
%:%440=164%:%
%:%441=165%:%
%:%442=165%:%
%:%443=166%:%
%:%449=166%:%
%:%454=167%:%
%:%459=168%:%
